\documentclass{article}
\usepackage{graphicx}

\begin{document}

\section{Project Plan}

\subsection{Project Overview}
The project aims to develop and maintain a web application named "Tweeter," inspired by the functionality and user experience of the popular social media platform Twitter. Tweeter will serve as a platform for users to engage in real-time conversations, share thoughts and updates, and connect with others in a dynamic online community. The application will replicate key features of Twitter, including tweeting, retweeting, liking, and following, while also introducing enhancements to improve user engagement and interaction.

\subsection{Deliverables}
\begin{itemize}
    \item Project plan document outlining project scope, goals, and timeline.
    \item Design artifacts including wireframes, mockups, and prototypes for key application screens.
    \item Implemented web application with core features and functionalities inspired by Twitter.
    \item Test documentation covering test plans, cases, and results for functional and non-functional testing.
    \item Deployment documentation detailing deployment process and configurations for production environment.
    \item Post-launch support plan outlining ongoing maintenance and support activities to ensure the long-term success of the application.
\end{itemize}

\subsection{Project Timeline}
\subsubsection{Planning and Requirements Gathering (Week 1-2)}
\begin{itemsize}
    \item Define project scope, goals, and requirements based on Twitter's functionality. Gather user feedback and insights to inform feature prioritization and development roadmap. Establish project team roles and responsibilities, and allocate resources accordingly.
    \item Create system requirements, user stories, and acceptance criteria for core features and functionalities. Develop a project plan, timeline, and budget based on resource estimates and deliverables.
    \item Conduct meetings to assign roles and responsibilities, establish communication channels, and align on project objectives and expectations.
    \item Review and finalize project plan, requirements, and timeline with team members to ensure alignment and commitment.
\end{itemsize}
\subsubsection{Design and Prototyping (Week 3-4)}
\begin{itemsize}
    \item Create wireframes, mockups, and prototypes for key application screens and user flows. Design user interfaces, navigation structures, and visual elements inspired by Twitter's design language. Review and iterate on design concepts based on stakeholder feedback and usability testing.
    \item Design database schema, data models, and API specifications to support core features and functionalities. Define data structures, relationships, and access patterns to ensure efficient data storage and retrieval.
    \item Develop design guidelines, style guides, and branding assets to maintain consistency and coherence across the application. Define typography, color palettes, iconography, and imagery to reflect the project's identity and messaging.
    \item Conduct design reviews, usability testing, and accessibility audits to validate design decisions and ensure alignment with user needs and expectations.
    \item Finalize design assets, prototypes, and specifications for handoff to development team. Document design decisions, rationale, and guidelines for future reference and maintenance.
\end{itemsize}
\subsubsection{Development and Implementation (Week 5-6)}
\beign{itemsize}
    \item Start with the basic styling of the application, focusing on the layout, typography, and color scheme. Implement responsive design and accessibility features to ensure compatibility with various devices and user needs.
    \item Document all the necessary components and their functionalities. Implement the core features of the application, including user authentication, tweet posting, interaction, and profile management. Develop reusable components and modules to streamline development and ensure consistency.
    \item Add mock data and test the application's functionality in a controlled environment. Conduct user testing and feedback sessions to identify usability issues, bugs, and performance bottlenecks. Iterate on design and implementation based on user feedback and testing results.
    \item Set up development environment and infrastructure, including backend services and database. Implement core features and functionalities of the application, focusing on tweet posting, interaction, and profile management. Conduct iterative development sprints and code reviews for quality assurance and feature completeness.
end{itemsize}
\subsubsection{Testing and Quality Assurance (Week 7)}
\beign{itemsize} 
\item Develop test plans, cases, and scripts for functional, performance, and security testing. Perform unit testing, integration testing, and user acceptance testing to identify and address defects and issues. Conduct usability testing and accessibility testing to ensure a seamless user experience for all users.
\subsubsection{Deployment and Launch (Week 8)}
\beign{itemsize}
\item Prepare deployment environment and configuration settings for staging and production servers. Deploy the application to production servers and cloud platforms, and conduct final testing and validation. Coordinate marketing and promotional activities to generate awareness and drive user adoption. 

\subsection{Software Engineering Methodology}
\begin{itemsize}
\item We will follow the Agile methodology for this project. Agile allows for iterative development, frequent feedback, and adaptability, which are essential for a dynamic project like this. By breaking down tasks into smaller, manageable units, we can deliver incremental value to stakeholders and respond quickly to changing requirements.
\item We will use Scrum as our Agile framework, with sprints lasting one week each. Scrum provides a structured approach to project management, with defined roles, ceremonies, and artifacts that help us stay organized and focused. Daily stand-ups, sprint planning, sprint reviews, and retrospectives will be conducted to ensure transparency, collaboration, and continuous improvement.
\item We will use Github Project Board as our project management tool to track tasks, user stories, and progress. Github Project Board provides a centralized platform for managing project activities, assigning tasks, and monitoring team performance. It also integrates with other tools and services to streamline development, testing, and deployment processes.
\item We will use GitHub as our version control system to manage code repositories, branches, and pull requests. GitHub provides collaboration features such as code reviews, issue tracking, and project boards that facilitate teamwork and code quality. It also integrates with CI/CD tools and services to automate testing, deployment, and monitoring.
\item We will use Discord as our communication tool to facilitate real-time messaging, file sharing, and collaboration. Discord provides channels for team discussions, announcements, and updates, as well as integrations with other tools and services for seamless workflow management. It also supports video calls, screen sharing, and notifications to keep team members connected and informed.
\subsection{Team Composition}
\subsubsection{Project Manager: Tessa Engelbrecht}
Tessa oversees the entire project from initiation to completion. She is responsible for planning, organizing, and coordinating all project activities. Tessa ensures that project goals are clearly defined, deadlines are met, and resources are allocated effectively. She also facilitates communication between team members, stakeholders, and management to ensure smooth project execution.
\subsubsection{Business Analyst: Kumbirai Shonhiwa}
Kumbirai gathers and analyzes business requirements, user needs, and market trends to identify opportunities for the project. He translates these requirements into actionable insights and specifications that guide the development process. Kumbirai works closely with stakeholders to ensure that the project meets business objectives and delivers value to the organization.
\subsubsection{UI/UX Engineer: Yashvitha Kanaparthy}
Yashvitha focuses on designing intuitive and engaging user interfaces and experiences for the application. She collaborates with stakeholders and designers to understand user requirements and preferences, then translates them into wireframes, mockups, and prototypes. Yashvitha ensures that the application's UI/UX aligns with best practices and enhances user satisfaction and usability.
\subsubsection{Designer: Kyle Marshall}
Kyle is responsible for creating visually appealing and consistent designs for the application. He uses design tools and principles to develop layouts, graphics, and visual elements that convey the project's branding and messaging effectively. Kyle works closely with the UI/UX engineer to ensure that designs are implemented seamlessly into the application.
\subsubsection{DevOps: Michael Chinyama}
Michael is responsible for automating and streamlining the development, deployment, and operations processes. He sets up and maintains the infrastructure, tools, and pipelines required for continuous integration, delivery, and monitoring of the application. Michael also collaborates with developers to optimize performance, scalability, and reliability of the application.
\subsubsection{Integration Engineer: Quintin D'hotman de Villiers}
Quintin specializes in integrating different software systems, modules, and components to ensure seamless communication and interoperability. He designs and implements APIs, middleware, and connectors that enable data exchange and workflow automation across the application stack. Quintin also troubleshoots and resolves integration issues to maintain system integrity.
\subsubsection{Testing Engineer: Alex Pretorius}
Alex is responsible for ensuring the quality and reliability of the application through comprehensive testing. He develops and executes test plans, cases, and scripts to identify defects, bugs, and performance issues. Alex works closely with developers to troubleshoot and resolve issues, ensuring that the application meets quality standards and user expectations.
\subsubsection{Database Engineer: Kamogelo Moeketse}
Kamogelo specializes in designing, implementing, and optimizing databases to store and retrieve data efficiently. He selects appropriate database technologies, schemas, and configurations based on project requirements and scalability needs. Kamogelo also ensures data integrity, security, and compliance with regulatory standards.
\subsubsection{Architect: Dhinaz Rangasamy}
Dhinaz is responsible for designing the overall architecture and technical framework of the application. He defines the system's structure, components, interfaces, and dependencies to ensure scalability, flexibility, and maintainability. Dhinaz also guides and mentors other team members on architectural best practices and patterns.
\subsubsection{Services Engineer: Dominique Da Silva, Given Chauke}
Dominique and Given focus on designing, implementing, and maintaining the backend services and APIs required for the application. They develop scalable, reliable, and secure microservices, serverless functions, and web APIs that support the application's functionality. Dominique and Given also monitor and optimize service performance and availability to meet SLAs and user expectations.

\subsection{Budget}
The budget for this project will be determined based on resource allocation, infrastructure costs, and other expenses incurred during development, testing, deployment, and maintenance phases.
Each person in the group will contribute a minimum of R1000 per hour towards the project. This will cover the costs of working on resource creation, management of the database, creating mock data and populating the database, and the costs of the domain and hosting.
The total budget for the project is estimated to be R100,000, with a breakdown of costs as follows:
\begin{itemize}
    \item Development and Implementation: R30,000
    \item Design and Prototyping: R20,000
    \item Testing and Quality Assurance: R10,000
    \item Deployment and Launch: R10,000
    \item Post-launch Support: R10,000
    \item Contingency: R20,000 
\end{itemize}
\subsection*{Conclusion}
The project plan outlines the scope, goals, timeline, and budget for developing the Tweeter web application. By following Agile methodologies, leveraging the skills and expertise of the project team, and adhering to best practices in software engineering, we aim to deliver a high-quality, user-centric application that meets stakeholder expectations and user needs. The project plan will be regularly reviewed and updated to ensure alignment with project objectives and milestones.
\end{document}